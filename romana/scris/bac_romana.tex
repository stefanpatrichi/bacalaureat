\documentclass[a4paper]{book}

\usepackage{graphicx}
\usepackage[left=1in,right=1in,top=1.25in,bottom=1.25in,headsep=30pt,heightrounded]{geometry}
\usepackage[romanian]{babel}
\usepackage{indentfirst}
\usepackage{hyperref}
\usepackage{fancyhdr}
\setlength{\headheight}{14.5pt}
\pagestyle{fancy}
\fancyhf{}
\fancyhead[LE,RO]{\thepage}
\fancyhead[LO]{\small\nouppercase{\rightorleftmark}}
\fancyhead[RE]{\small\nouppercase{\leftmark}}
\renewcommand\chaptermark[1]{\markboth{\thechapter\hspace{0.25cm} #1}{}}
\renewcommand\sectionmark[1]{\markright{\thesection\hspace{0.4cm}#1}{}}
\renewcommand{\headrulewidth}{0pt}

\makeatletter
\newcommand{\rightorleftmark}{%
\begingroup\protected@edef\x{\rightmark}%
\ifx\x\@empty
 \endgroup\leftmark
\else
 \endgroup\rightmark
\fi}
\makeatother

\begin{document}
\begin{titlepage}
\frontmatter
    \begin{center}
        \phantom{\tiny a}
   { \Huge
    \vfill
    \textbf{Bacalaureat\\Limba și literatura română\\Proba scrisă}
    \vfill
    }
    \end{center}
    
    \begin{flushright}
    \large
dr. Anca Evelina \textsc{Cîrligeanu} \\
Ștefan \textsc{Patrichi}
\end{flushright}
\end{titlepage}

\tableofcontents

\mainmatter
\chapter{Concepte operaționale}
\section{Genul epic}
\section{Genul liric}
\section{Genul dramatic}

\chapter{Eseuri}
\section{„Alexandru Lăpușneanul” de Costache Negruzzi}
\section{„Povestea lui Harap-Alb” de Ion Creangă}
% particularități stilistice ale narațiunii crengiste
\section{„Moara cu noroc” de Ioan Slavici}
\section{„Ion” de Liviu Rebreanu}
\section{„Baltagul” de Mihail Sadoveanu}
\section{„Ultima noapte de dragoste, întâia noapte de război” de Camil Petrescu}
\section{„Maitreyi” de Mircea Eliade}
\section{„Enigma Otiliei” de George Călinescu}
\section{„Moromeții” de Marin Preda}
\section{„O scrisoare pierdută” de Ion Luca Caragiale}
\section{„Iona” de Marin Sorescu}
\section{„Luceafărul” de Mihai Eminescu}
\section{„Plumb” de George Bacovia}
\section{„Eu nu strivesc corola de minuni a lumii” de Lucian Blaga}
\section{„Testament” de Tudor Arghezi}
\section{„Joc secund” de Ion Barbu}
\section{„Aci sosi pe vremuri” de Ion Pillat}
\section{„Leoaică tânără, iubirea” de Nichita Stănescu}
\section{„Levantul” de Mircea Cărtărescu}
\end{document}